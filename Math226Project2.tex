\documentclass{article}
\usepackage{mathtools}
\usepackage[margin=0.5in]{geometry}
\begin{document}

\begin{text}
Stephen (Vinnie) Magro \& Nick Entin \\
\indent MATH 226 Project \#2 - Distance from a point to line or to a plane \\
\newline
\newline
\end{text}
\begin{text}
\indent Problem 1 \\
\end{text}

% begin 1.1
\begin{text}
1) Consider the plane $P$ of equation $x + y + z - 1 = 0$ and the point $Q = (2, 1, 2)$
\newline
\end{text}

\begin{text}
\indent a) Find the equation of the line orthogonal to $P$ going through $Q$.
\newline\newline
\indent\indent\indent Given the equation of $P$, we can define the normal vector to $P$:
\end{text}

\begin{align*}
\vec{N}=<1,1,1>
\end{align*}

\begin{text}
\indent\indent By definition, the line orthogonal to $P$ must be parallel to it's normal vector and pass through $Q$:
\end{text}

\begin{align*}
\vec{L}(t) &= \vec{Q} + \vec{N}t \\
           &= < 2, 1, 2 > + < 1, 1, 1 >t \\
           &= < 2 + t, 1 + t, 2 + t >
\end{align*}

\begin{text}
\indent b) Find the coordinates of the intersection point of that line with $P$.
\newline\newline
\indent\indent\indent First, we must solve for $t$ where it fulfills the equation of the plane $P$.
\end{text}

\begin{align*}
x + y + z - 1 &= 0 \\
2 + t + 1 + t + 2 + t -1 &= 0 \\
3t + 4 &= 0 \\
t &= -\frac{4}{3}
\end {align*}

\begin{text}
\indent\indent We can then plug our new value of $t$ into our orthogonal line to find the point of intersection.
\end{text}

\begin{align*}
I &= \vec{L}(-\frac{4}{3}) \\
  &= (2 - \frac{4}{3}, 1 - \frac{4}{3}, 2 - \frac{4}{3}) \\
  &= (\frac{2}{3}, -\frac{1}{3}, \frac{2}{3})
\end{align*}

\begin{text}
\indent c) The distance between a point and a plane is the distance between the point and its orthogonal projection on the \\
\indent\indent\indent plane.  Use b) to compute the distance between $Q$ and $P$.
\end{text}

\begin{align*}
d &= \sqrt{(x_2 - x_1)^2 + (y_2 - y_1)^2 + (z_2 - z_1)^2} \\
&= \sqrt{(2 - \frac{2}{3})^2 + (1 + \frac{1}{3})^2 + (2 - \frac{2}{3})^2} \\
&= \sqrt{\frac{16}{9} + \frac{16}{9} + \frac{16}{9}} \\
&= \sqrt{\frac{16}{3}} = \frac{4}{\sqrt{3}} = \frac{4\sqrt{3}}{3}
\end{align*}
% end 1.1

\newpage

% begin 1.2
\begin{text}
2) Given a plane $P$ of equation $ax + by + cz + d = 0$ and a point $Q = (x_0, y_0, z_0)$. Let denote $\vec{N} = (a, b, c)$ a normal \\
\indent\indent vector to $P$.  Show that the distance between $Q$ and $P$ is given by:
$$
\frac{|ax_0 + by_0 + cz_0 + d|}{||\vec{N}||}
$$
\indent\indent Check your answer from 1)c) using the formula above. \\
\newline
\indent\indent We define $\vec{L}$ as the line orthogonal to $P$ and passing through $Q$.
\end{text}

$$
\vec{L}(t) = <x_0 + at, y_0 + bt, z_0 + ct>
$$

\begin{text}
\indent We then plug $\vec{L}$ into the equation of $P$.
\end{text}

\begin{align*}
ax + by + cz + d &= 0 \\
a(x_0 + at) + b(y_0 + bt) + c(z_0 + ct) + d &= 0 \\
ax_0 + a^2t + by_0 + b^2t + cz_0 + c^2t + d &= 0 \\
a^2t + b^2t + c^2t + ax_0 + by_0 + cz_0 + d &= 0
\end{align*}

\begin{text}
\indent From this equation, we solve for $t$ to get:
$$
t = -\frac{ax_0 + by_0 + cz_0 + d}{a^2 + b^2 + c^2}
$$
\indent\indent We can now find the point on $P$ where $\vec{L}$ originates:
\end{text}

\begin{align*}
I &= \vec{L}(-\frac{ax_0 + by_0 + cz_0 + d}{a^2 + b^2 + c^2}) \\
  &= (x_0+a(-\frac{ax_0 + by_0 + cz_0 + d}{a^2 + b^2 + c^2}), y_0+b(-\frac{ax_0 + by_0 + cz_0 + d}{a^2 + b^2 + c^2}), z_0+c(-\frac{ax_0 + by_0 + cz_0 + d}{a^2 + b^2 + c^2})) \\
  &= (x_0-a\frac{ax_0 + by_0 + cz_0 + d}{a^2 + b^2 + c^2}, y_0-b\frac{ax_0 + by_0 + cz_0 + d}{a^2 + b^2 + c^2}, z_0-c\frac{ax_0 + by_0 + cz_0 + d}{a^2 + b^2 + c^2})
\end{align*}

\begin{text}
\indent Next, we find the distance $d$ from $I$ to $Q$: 
\end{text}

\begin{align*}
d &= \sqrt{(x_0 - x_0 + a\frac{ax_0 + by_0 + cz_0 + d}{a^2 + b^2 + c^2})^2 + (y_0 - y_0 + b\frac{ax_0 + by_0 + cz_0 + d}{a^2 + b^2 + c^2})^2 + (z_0 - z_0 + c\frac{ax_0 + by_0 + cz_0 + d}{a^2 + b^2 + c^2})^2} \\
&= \sqrt{\frac{a^2(ax_0+by_0+cz_0+d)^2 + b^2(ax_0+by_0+cz_0+d)^2 + c^2(ax_0+by_0+cz_0+d)^2}{(a^2 + b^2 + c^2)^2}} \\
&= \sqrt{\frac{(a^2 + b^2 + c^2)(ax_0+by+0+cz+0+d)^2}{(a^2 + b^2 + c^2)^2}} \\
&= \sqrt{\frac{(ax_0 + by_0 + cz+0 +d)^2}{a^2 + b^2 + c^2}} \\
&= \frac{|ax_0 + by_0 + cz_0 + d|}{\sqrt{a^2+b^2+c^2}}
\end{align*}

\begin{text}
\indent Given $\vec{N}=<a,b,c>$, we can show that: \\
$$
||\vec{N}|| = \sqrt{a^2+b^2+c^2}
$$
\indent\indent Upon replacing this value in $d$ we finally prove: \\
$$
d = \frac{|ax_0 + by_0 + cz_0 + d|}{||\vec{N}||}
$$
\end{text}

% going to next page to kept following section together
\newpage

\begin{text}
\indent Now, let's check our answer from 1)c):
\end{text}

\begin{align*}
d &= \frac{| 2 + 1 + 2 - 1 |}{\sqrt{1^2 + 1^2 + 1^2}} \\
&= \frac{| 4 |}{\sqrt{3}} = \frac{4\sqrt{3}}{3}
\end{align*}

% end 1.2

% begin 2.1

\begin{text}
Problem 2 \\

1) Consider the line $\vec{L}$ and point $Q$ as defined below:
\[\vec{L}(t) = \left\{
  \begin{array}{lr}
    x &= t + 2 \\
    y &= t - 1 \\
    z &= t + 1
  \end{array}
\right.
\;\;\;\;\;
Q = (2,1,2)
\]

\indent\indent Find $H$ on $\vec{L}$ such that $\vec{HQ} \perp \vec{L}$
\end{text}

\begin{align*}
\text{Let } \vec{u} &= \text{direction vector of } \vec{L} \\
&= <1,1,1>
\end{align*}

\begin{text}
\indent a) Find the coordinates of the point $H$ on the line $\vec{L}$ such that $\vec{HQ}$ is orthogonal to $\vec{L}$.
\end{text}

\begin{align*}
\vec{QH} &= <2, -1, 1> + <1,1,1>t - <2,1,2> \\
         &= <t, t-2, t-1>
\end{align*}

\begin{text}
\indent\indent Since $\vec{QH} \perp \vec{L}$ and $\vec{u}$ is the direction vector of $\vec{L}$, then $\vec{QH} \perp \vec{u}$ as well. \\ \\
\indent\indent\indent Using the knowledge that for any two orthogonal vectors, the dot product of the two vectors will equal $0$, we \\
\indent\indent\indent can find the value of $t$ for which $H = \vec{L}(t)$:
\end{text}

\begin{align*}
\vec{QH} \cdot \vec{u} &= 0  \text{ (orthogonal)}\\
t + t - 2 + t -1 &= 0 \\
3t - 3 &= 0 \\
3t &= 3 \\
t &= 1
\end{align*}

\begin{align*}
H &= (1+2, 1-1, 1+1) \\
H &= (3, 0, 2)
\end{align*}

\begin{text}
\indent b) The distance between a point and a line is the distance between the point and its orthogonal projection onto the \\
\indent\indent\indent line.  Use b) to compute the distance between $Q$ and $L$.
\end{text}

\begin{align*}
d &= \sqrt{(3-2)^2 + (0-1)^2 + (2-2)^2} \\
  &= \sqrt{1^2 + (-1)^2 + 0^2} \\
  &= \sqrt{2}
\end{align*}

% going to next page to keep following section together
\newpage

\begin{text}
2) Given a line $L$ and point $Q$ as defined below:
\end{text}

\[\vec{L}(t) = \left\{
  \begin{array}{lr}
    x &= \alpha + at \\
    y &= \beta + bt \\
    z &= \gamma + ct
  \end{array}
\right.
\;\;\;
Q = (x_0, y_0, z_0)
\]

\begin{align*}
\text{Let } \vec{v} &= \text{direction vector of } \vec{L} \\
&= <a, b, c>
\end{align*}

\begin{text}
\indent Find a formula for the distance between $Q$ and $L$. \\ \\
\indent\indent Let $H$ be the point on $\vec{L}$ where $\vec{QH} \perp \vec{v}$
\end{text}

\begin{align*}
\vec{QH} &= <\alpha, \beta, \gamma> + t<a, b, c> - <x_0, y_0, z_0> \\
\vec{QH} &= <at + \alpha - x_0, bt + \beta - y_0, ct + \gamma - z_0>
\end{align*}
\begin{align*}
\vec{QH} \cdot \vec{v} &= 0  \text{ (orthogonal)} \\
a(at + \alpha - x_0) + b(bt + \beta - y_0) + c(ct + \gamma - z_0) &= 0 \\
at^2 + a\alpha - ax_0 + bt^2 + b\beta - by_0 + ct^2 + c\gamma - cz_0 &= 0 \\
t(a^2 + b^2 + c^2) + a(\alpha - x_0) + b(\beta - y_0) + c(\gamma - z_0) &= 0 \\
\\
t = \frac{-a(\alpha - x_0) - b(\beta - y_0) - c(\gamma - z_0)}{a^2+b^2+c^2}
\end{align*}
\begin{equation*}
\begin{multlined}
H = \vec{L}(t) = (\alpha + a(\frac{-a(\alpha - x_0) - b(\beta - y_0) - c(\gamma - z_0)}{a^2+b^2+c^2}), \beta + b(\frac{-a(\alpha - x_0) - b(\beta - y_0) - c(\gamma - z_0)}{a^2+b^2+c^2}), \\\gamma + c(\frac{-a(\alpha - x_0) - b(\beta - y_0) - c(\gamma - z_0)}{a^2+b^2+c^2})
\end{multlined}
\end{equation*}

\begin{alignat*}{3}
\alpha &= 2 \quad & \beta &= -1 \quad & \gamma &= 1 \\
a &= 1    \quad & b &= 1    \quad & c &= 1\\
x_0 &= 2    \quad & y_0 &= 1  \quad & z_0 &= 2\\
\end{alignat*}
\begin{align*}
\text{Let } k &= \frac{-a(\alpha - x_0) - b(\beta - y_0) - c(\gamma - z_0)}{a^2+b^2+c^2} \\
\text{For variables as defined above:} \\
k &= \frac{-1(2-2) - 1(-1-1) - 1(1-2)}{1^2 + 1^2 + 1^2} \\
k &= 1
\end{align*}

Check 1.b
\begin{align*}
d &= \sqrt{(\alpha + a(k) - x_0)^2 + (\beta + b(k) - y_0)^2 + (\gamma + c(k) - z_0)^2} \\
&= \sqrt{(2 + 1(1) - 2)^2 + (-1 + 1(1) - 1)^2 + (1 + 1(1) - 2)^2} \\
&= \sqrt{1 + 1 + 0} \\
d &= \sqrt{2} \quad \text{ (Matches answer in 1.b)}
\end{align*}


\end{document}